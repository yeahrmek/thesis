\chapter{Applications}
\label{chap:applications}

This chapter is dedicated to applications of the developed techniques
in several different problems.
We demonstrate how the described approaches for GP regression
and kernel approximation can be used in a tensor completion
problem,
density estimate and simultatineous localization and mapping (SLAM).
This is a diverse set of problems: density estimation is a key problem
in statistics, SLAM is one of the main problems in robotics,
while tensor completion is a very general problem that can be encountered
in computer vision, signal processing, machine learning and many others.
All of them require large-scale or online methods and can benefit
from using GP or kernel based approaches.
In this chapter we show how large-scale GP methods can be incroporated into
existing approaches.
The resulting techniques provides state-of-the-art results both in terms of
accuracy and computational complextiy.

Section \ref{sec:tensor_completion_using_gp} describes our approach for
tensor completion using Gaussian Processes.
Our contributions here are the following:
\begin{itemize}
    \item We consider the case when the tensor is generated by some smooth function.
    Using this assumption we propose an initialization approach that
    can be used with a wide range of tensor completion techniques.
    \item We demonstrate empirically on real-world problems that different optimization methods
    for tensor completion benefit from using the proposed initialization.
    It takes into account assumption about tensor generating function and, therefore, allows to increase generalization power.
\end{itemize}
In section \ref{sec:score_matching} we develop randomized feature maps based approach for density estimation.
We seek our solution in kernel exponential family of distributions with the denoising score matching loss function.
Our main results are as follows:
\begin{itemize}
    \item We derived an analytical solution for the described setup.
    \item We show that our solution has implicit regularization in constrast
    to other approaches that usually add special regulaarization term based on higher
    order derivatives of the probability density function.
    \item Finally, we demonstrate the benefits of the proposed approach on a set of different
    benchmarks and compare it with other techniques.
\end{itemize}
The last section \ref{sec:SLAM} is devoted to GP models with random feature approximations
in SLAM problem.
The GP model is used here to model state of the robot depending on time.
We contribute to this field by
\begin{itemize}
    \item Developing an approach based on random features for time-continuous SLAM.
    \item Compared to other state-of-the-art aproaches our method is more accurate
    in case of noisy observations because we do not assume the sparse structure of the
    inverse covariance matrix. We demonstrate it on a set of synthetic benchmarks
    as well as real-world problems.
\end{itemize}